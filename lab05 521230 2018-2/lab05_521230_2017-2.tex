\documentclass[letterpaper,11pt]{article}
\usepackage[spanish,activeacute]{babel}
\decimalpoint
\usepackage{amsmath,amsfonts,amssymb}
\usepackage{fancyhdr}
\usepackage{fancyvrb}
\usepackage{graphicx}
\usepackage{enumerate}
\usepackage{multirow}
\usepackage{multicol}
\usepackage{longtable}
\usepackage{hyperref}
\hypersetup{
    colorlinks=true,       % false: boxed links; true: colored links
    linkcolor=red,          % color of internal links (change box color with linkbordercolor)
}
\usepackage[numbered,framed]{matlab-prettifier}
\lstMakeShortInline"
\lstset{
  style              = Matlab-editor,
  escapechar         = ",
  mlshowsectionrules = true,
}

\setlength{\oddsidemargin}{-0.5cm}
\setlength{\evensidemargin}{0cm}
\setlength{\textwidth}{17.5cm}
\setlength{\textheight}{24cm}
\setlength{\topmargin}{-1.7cm}

\title{lab05 521230 2018-1}

\font\bff=cmbx10 at 10truept
\font\lg=cmdunh10 at 10truept
\font\bl=cmss10 at 10truept

\newcommand{\matlab}{{\sc Matlab} }

\newcommand{\header}{\noindent%
{\lg UNIVERSIDAD DE CONCEPCI\'ON}\hfill
\vskip-4truept
\noindent{\bff FACULTAD DE CIENCIAS}\hfill
\vskip-4truept
\noindent{\bff F\'ISICAS Y MATEM\'ATICAS}\hfill
\vskip-4truept
\noindent{\bl DEPARTAMENTO DE INGENIER\'IA MATEM\'ATICA}\hfill
\vskip4truept\hrule\hrule\vskip4truept
\par
}

\begin{document}
\header
\vspace{0.7cm}
\begin{center}
\textbf{\small C\'alculo Num\'erico (521230) - Laboratorio 5}\\
\vspace{0.1cm}
\textbf{\Large INTEGRACI\'ON NUM\'ERICA EN MATLAB}
\vspace{0.7cm}
\end{center}

\section{Reglas de Newton--Cotes}
Descargue la funci\'on de \matlab 
\href{ftp://ftp.ing-mat.udec.cl/pub/ing-mat/asignaturas/521230/ejercicios/2018-1/trap.m}{trap.m} 
esta funci\'on consiste en una implementaci\'on del m\'etodo del trapecio. Utilice la funci\'on \texttt{help} para aprender a usar la funci\'on \texttt{trap}.

Mediante la funci\'on \texttt{trap} estime las integrales definidas
$$
\displaystyle
\int_{-\pi}^{\pi} \cos(x+1)\, \mathrm{d}x, \quad
\int_{-4}^{7} (2x^2-x+1) \, \mathrm{d}x, \quad
\int_{-1}^{1} e^{2x+1} \sen(2x+1)\, \mathrm{d}x. \quad
$$


El c\'odigo disponible en 
\href{ftp://ftp.ing-mat.udec.cl/pub/ing-mat/asignaturas/521230/ejercicios/2018-1/cuadratura1.m}{cuadratura1.m} 
genera una animaci\'on que muestra la construcci\'on de una regla de cuadratura compuesta para una funci\'on dada. Descargue este programa y comente qu\'e hace cada l\'inea del c\'odigo.


\textbf{Ejercicio}

Cree un rutero de \matlab que permita aproximar, usando las reglas del punto medio y trapecio, la integral definida
$$
\displaystyle
\int_{-2}^2 \left|\cos(3x)+\frac{\sen(x^2)}{10}\right|\,\mathrm{d}x
$$
y que le permita completar la siguiente tabla
\begin{center}
\begin{tabular}{c|c|c}
Cant. nodos & Punto medio & Trapecio \\
\hline
2& & \\
3 & & \\
4 & & \\
5 & & \\
6 & & \\
7 & & \\
8 & & \\
9 & & \\
\end{tabular}
\end{center}
de los valores aproximados para esta integral usando estos m\'etodos para reglas compuestas con distintas cantidades de nodos.

\textquestiondown Qu\'e observa en los digitos obtenidos?, \textquestiondown C\'ual de los dos m\'etodos funciona mejor en este caso?.

\subsection{Generalizaci\'on}
Estas reglas provienen de la aproximaci\'on de una funci\'on por su interpolante en los puntos $\{(x_i,f(x_i))\}_{i=0}^n$
$$
\displaystyle
f(x)\approx \sum_{i=0}^n f(x_i) \ell_i(x)=\sum_{i=0}^n f(x_i) \prod_{\substack{k=0\\k\neq i}}^n \frac{x-x_k}{x_i-x_k},
$$
de donde surge la idea de aproximar la integral definida mediante la integral de la interpolante, a la que llamamos \emph{cuadratura}:
$$
\displaystyle
\int_{x_0}^{x_n}
f(x)\,\mathrm{d}x \approx 
\int_{x_0}^{x_n}
\sum_{i=0}^n
f(x_i) \ell_i(x)\,\mathrm{d}x 
=
\sum_{i=0}^n
f(x_i) 
\int_{x_0}^{x_n} \ell_i(x)\,\mathrm{d}x.
$$
As\'i, los \emph{pesos} de estas reglas son
$$
(\forall\,i\in\{0,\dotsc,n\}) \quad w_i = \int_{x_0}^{x_n} \ell_i(x)\,\mathrm{d}x.
$$

En el caso \textbf{muy particular} en el que los nodos $\{x_i\}_{i=0}^n$ son equiespaciados con tama\~no de paso $h$, entonces mediante el cambio de variable
$
x=x_0+h\,t, 
$
se tiene que
$$
w_i
=	\int_{x_0}^{x_n} \ell_i(x)\,\mathrm{d}x 
=	\int_{x_0}^{x_n}
	\prod_{\substack{k=0\\k\neq i}}^n 
    	\frac{x-x_k}{x_i-x_k}\,\mathrm{d}x
=	\int_{0}^{n}
	\prod_{\substack{k=0\\k\neq i}}^n 
    	\frac{t-k}{i-k}\,h\,\mathrm{d}t   
=	h \int_{0}^{n}
	\prod_{\substack{k=0\\k\neq i}}^n 
    	\frac{t-k}{i-k}\,\mathrm{d}t.
$$
Observe estas \'ultimas integrales, a saber,
 $$
\alpha_i= \int_{0}^{n}
	\prod_{\stackrel{k=0}{k\neq i}}^n 
    	\frac{t-k}{i-k}\,\mathrm{d}t,
 $$
son independientes del intervalo de integraci\'on. Para el caso de $n=2$, tenemos que
 $$
 \begin{array}{cllll}
 \alpha_0 
 &\displaystyle 
 =\int_0^2 \frac{t-1}{0-1} \, \frac{t-2}{0-2}\,\mathrm{d}t
 &\displaystyle
 =\frac{1}{2}\int_0^2 (t^2-3t+2)\,\mathrm{d}t
 &
 =\frac{1}{2}\left(\frac{8}{3}-\frac{12}{2}+4\right)=\frac{1}{3}
\\ \\
\alpha_1 
 &\displaystyle 
 =\int_0^2 \frac{t-0}{1-0} \, \frac{t-2}{1-2}\,\mathrm{d}t
 &\displaystyle
 =-\int_0^2 (t^2-2t)\,\mathrm{d}t
 &
 =-\left(\frac{8}{3}-4\right)=\frac{4}{3}
 \\ \\
\alpha_2 
 &\displaystyle 
 =\int_0^2 \frac{t-0}{2-0} \, \frac{t-1}{2-1}\,\mathrm{d}t
 &\displaystyle
 =\frac{1}{2}\int_0^2 (t^2-t)\,\mathrm{d}t
 &
=\frac{1}{2}\left(\frac{8}{3}-\frac{4}{2}\right) =\frac{1}{3}\\
 \end{array}
 $$
 con la cual se genera la \emph{regla de Simpson}:
 $$
 \displaystyle
 \int_{x_0}^{x_n} f(x)dx \approx \frac{h}{3} f(x_0)+\frac{4h}{3} f(x_1)+\frac{h}{3} f(x_2).
 $$
Aqu\'i, $h = x_1-x_0 = x_2-x_1$. Los valores exactos de estas los pesos de estas reglas de cuadratura, para distintas cantidades de nodos, est\'an disponibles en el archivo
 \href{ftp://ftp.ing-mat.udec.cl/pub/ing-mat/asignaturas/521230/ejercicios/2018-1/pesosnc.mat}{pesosnc.mat}

\subsection{Ejercicios.}
\begin{enumerate}
\item 
Calcule anal\'iticamente los coeficientes $\alpha_0,\alpha_1$ de una regla de Newton--Cotes con dos nodos. \textquestiondown Reconoce esta regla?.

\item
Las reglas de Newton--Cotes de 4 y 5 nodos son conocidas como de Simpson-$3/8$ y de Milne, respectivamente.
Calcule anal\'iticamente los coeficientes asociados a estas reglas.

\item Aproxime usando la regla de Simpson las siguientes integrales
\begin{multicols}{2}
\begin{enumerate}
\item $\int_0^\pi \cos(2\pi x)\,\mathrm{d}x$
\item $\int_{-1}^1 e^{x+1}\,\mathrm{d}x$
\item $\int_{-\pi}^\pi \sen(2\pi x)\,\mathrm{d}x$
\item $\int_{-10}^{10} (3x^2+2x+1)\,\mathrm{d}x$
\end{enumerate}
\end{multicols}
A continuaci\'on calcule anal\'iticamente el valor de cada una de estas integrales y calcule el error cometido en cada caso.

\item Aproxime el valor de la integral
$$
\int_{-1}^1 \frac{\cos(30\pi x)}{2x^2+1}\, dx
$$
usando una sucesi\'on de reglas compuestas de Simpson. Considere nodos equiespaciados con
$$
h=0.5, \quad h=0.25,\quad h=0.1 \quad\text{y}\quad h=0.05
$$
\item Escriba un rutero de \matlab que grafique a la funci\'on $f$ definida por
$$
f(x)=\frac{\cos(x)}{x^4+1}
$$
en el intervalo $[-5,5]$. Agregue al gr\'afico los $5$ puntos que subyacen a la regla de cuadratura de Newton--Cotes de 5 puntos\footnote{Recuerde que el interpolante de estos cinco puntos es un polinomio de grado 4 y que los cinco puntos deben ser equiespaciados en el eje de las abcisas.}.

Luego grafique el polinomio interpolante en cuesti\'on y calcule usando la regla mencionada la aproximaci\'on de esta integral.

\item Modifique el programa anterior para que se calculen las integrales aproximadas usando las reglas disponibles en
 \href{ftp://ftp.ing-mat.udec.cl/pub/ing-mat/asignaturas/521230/ejercicios/2018-1/pesosnc.mat}{pesosnc.mat} . 
\textquestiondown Qu\'e observa con las convergencias de estas aproximaciones?.
\end{enumerate}

\section{Cuadraturas de Gauss}

Si bien las reglas de Newton--Cotes se pueden definir para una cantidad arbitraria de puntos, en la pr\'actica se observan errores significativos a medida que se aumenta la cantidad de puntos, los que pueden atribuirse al fen\'omeno de Runge.
Por esto, no suelen usarse reglas de Newton--Cotes compuestas cuya contraparte b\'asica (en un solo intervalo) involucre m\'as de 7 nodos.

Las reglas de cuadratura Gaussiana, permiten aumentar arbitrariamente la cantidad de nodos manteniendo la convergencia de la cuadratura y en general suelen tener una mejor aproximaci\'on que las reglas de Newton--Cotes con el mismo costo computacional.

La idea de estas cuadraturas es elegir los mejores nodos $\{x_i\}_{i=0}^n$ de modo que la regla de cuadratura resultante
$$
\displaystyle
\int_{-1}^1 f(x)\,\mathrm{d}x \approx w_1 f(x_1) +\cdots+ w_n f(x_n)
$$
sea exacta para polinomios del mayor grado posible.

En el archivo 
 \href{ftp://ftp.ing-mat.udec.cl/pub/ing-mat/asignaturas/521230/ejercicios/2018-1/gauss25.mat}{gauss25.mat}
se encuentran los pesos $w_i$ y nodos $x_i$ de las primeras 25 reglas de cuadratura Gaussiana.

\subsection{Ejercicios}
\begin{enumerate}
	\item Calcule anal\'iticamente la integral
    $$
    \int_{-1}^1 (x^7+2x-1) \,\mathrm{d}x
    $$
    y luego en un programa de \matlab h\'agalo usando las primeras 10 reglas de cuadratura de Gauss Calcule adem\'as el error de integraci\'on en cada una de estas reglas.
    
	\item En un rutero de \matlab cree una matriz que contenga los valores aproximados al estimar las integrales
    $$
    \displaystyle
    \int_{-1}^1 x(x-1)(x+1)\, \mathrm{d}x,
    \quad
    \int_{-\pi}^\pi \frac{\cos(10x)}{|x|+1}\, \mathrm{d}x,
    \quad
    \int_{0}^1 \frac{e^{2x}}{2x+1}\, \mathrm{d}x.
    $$
con las reglas de Gauss. \textquestiondown Qu\'e observa?.

Recuerde que debe hacer un cambio de variable si el intervalo no de integraci\'on no es $[-1,1]$.
\end{enumerate}


\section{Funciones propias de \matlab}
\matlab tiene dos funciones \texttt{quad} y \texttt{quadl} que permiten calcular de manera autom\'atica integrales 
de funciones suaves con error menor que una tolerancia dada. 

La funci\'on \texttt{quad} es un m\'etodo de bajo
orden basado en la regla de Simpson, mientras que \texttt{quadl} es un m\'etodo de alto orden.

\begin{enumerate}
\item Utilice el comando \texttt{help} de \matlab\, para conocer la sintaxis de estos comandos.
\item Aproxime mediante \texttt{quad} y \texttt{quadl} las integrales 
$$
\displaystyle
\int_{-1}^1 e^{-x^2}\,\mathrm{d}x \quad \text{y} \quad \int_0^1 \sqrt{x}\,\mathrm{d}x
$$
con la tolerancia por defecto que asumen los comandos y despu\'es  especificando que el error sea menor que $10^{-6}$.
\end{enumerate}

\section{Ejercicios}
\begin{enumerate}
\item
\begin{enumerate}
\item Escriba en un archivo \texttt{trap.m} un programa tipo \texttt{function} que clacule la aproximaci\'on de la integral de una funci\'on dada en un intervalo gen\'erico $[a,\,b]$ por la regla de los trapecios con $N$ subintervalos.

\item Testee su programa con las siguientes integrales y $N=10, 20, 40, 80$:

$$\int_0^3 x^2\,\mathrm{d}x\qquad\int_{-1}^1 e^{-x^2}\,\mathrm{d}x\qquad\int_1^2 \log(x)\,\mathrm{d}x\qquad\int_0^1 \sqrt{x}\,\mathrm{d}x$$

\item Haga un programa semejante para la regla de Simpson. Gu\'ardelo en un archivo \texttt{simpson.m} y pru\'ebelo con las mismas funciones.
\end{enumerate}

\item Escriba un programa tipo \texttt{function} en \matlab que reciba como entrada una funci\'on de 2 variables $f(x,y)$, los valores $a$, $b$, $c$, $d$ y $N$ y cuya salida sea el valor de la integral:
$$
\int_a^b\int_c^d f(x,y)\,\mathrm{d}y\,\mathrm{d}x
$$
donde la integral con respecto a $y$ sea resuelta mediante la regla de los trapecios y la integral con respecto a $x$ sea resuelta con la regla de Simpson.

\item El fin de este ejercicio es verificar experimentalmente que el error del m\'etodo de los trapecios
aplicado a un integrando con derivadas acotadas en el intervalo de integraci\'on es de orden $\mathcal{O}(h^2)$.

\begin{enumerate}
\item Haga un programa que calcule la integral $\int_0^1e^{-x}\,\mathrm{d}x$ por la regla de los trapecios con $N =
10,20,30,\ldots,100$ subintervalos y almacene los errores respectivos. En este caso para calcular el error utilice el valor verdadero de la integral, el cual puede calcularse exactamente.
\item Grafique en escala logar\'itmica estos errores versus $N$ y la funci\'on $f(N) = (1/N)^{2}$.\\
\textbf{Sugerencia:} para graficar en escala logar\'itmica utilice el comando \verb+loglog+ en lugar de \verb+plot+;
la sintaxis de ambos comandos es la misma.
\item Explique por qu\'e el gr\'afico anterior muestra que el error de la regla de los trapecios es en este
caso $\mathcal{O}(h^2)$.
\end{enumerate}

\item El fin de este ejercicio es verificar experimentalmente que el error del m\'etodo de los trapecios aplicado a un integrando con derivadas no acotadas en el intervalo de integraci\'on no es de orden $\mathcal{O}(h^2)$.
\begin{enumerate}
\item Repita el ejercicio anterior con la integral  $\int_0^1\sqrt{x}\,\mathrm{d}x$.
\item Para estimar el orden de convergencia del m\'etodo en este caso, determine por cuadrados
m\'inimos las constantes $C$ y $\alpha$ que mejor ajustan los errores mediante el modelo
$$\textrm{error}(h) \approx Ch^{\alpha}.$$
\item[5.3] Grafique en escala logar\'itmica los errores versus $N$ y la funci\'on $f(N) = (1/N)^{\alpha}$.
\end{enumerate}

\item
\begin{enumerate}
\item Encuentre el \'area entre las curvas $(x,f(x))$ y $(x,g(x))$ con
$$
f(x)=e^{x-x^2},\quad g(x) = \arctan x^2, \quad x \in [-3,\,3].
$$
{\bf Indicaci\'on:} Grafique las funciones en $[-3,\,3]$ y calcule el \'area entre las curvas utilizando \verb+quad+ y \verb+fzero+ de manera adecuada.

\item Encuentre todos los valores de $x \in [-3,\,3]$ tales que
$$
\int_0^x\sin t^2 \, \mathrm{d}t=\frac{\sqrt{\pi}}4.
$$
Utilice para ello \verb+fzero+ y \verb+quad+ de manera adecuada.
\end{enumerate}


\item 
Una \textbf{Serie de Fourier} para $f$ una funci\'on definida en el intervalo $[a,b]$ es una serie de la forma
$$SF\{f(x)\}=\dfrac{a_0}{2}+\sum_{n=1}^{\infty}\left(a_n\cos\left(\dfrac{2\pi nx}{b-a}\right)+b_n\sen\left(\dfrac{2\pi nx}{b-a}\right)\right),
$$
con
$$
a_0=\dfrac{2}{b-a}\int_{a}^{b}f(x)\,\mathrm{d}x,\quad a_n=\dfrac{2}{b-a}\int_{a}^{b}f(x)\cos\left(\dfrac{2\pi nx}{b-a}\right)\,\mathrm{d}x \quad \textrm{y}\quad b_n=\dfrac{2}{b-a}\int_{a}^{b}f(x)\sen\left(\dfrac{2\pi nx}{b-a}\right)\,\mathrm{d}x, 
$$
la cual converge puntualmente a $f$, cuando $f$ es continua por pedazos en $[a, b]$, es decir $SF\{f(x)\}=f(x)$.

Para $N\in\mathbb{N}$, se define la \textbf{Suma Parcial de Fourier} como

$$
SF_N\{f(x)\}=\dfrac{a_0}{2}+\sum_{n=1}^{N}\left(a_n\cos\left(\dfrac{2\pi nx}{b-a}\right)+b_n\sen\left(\dfrac{2\pi nx}{b-a}\right)\right).
$$

\begin{enumerate}
\item Escriba una funcion en \matlab cuyas entradas sean $f$, el intervalo $[a,b]$ y $N$. Esta funci\'on debe dibujar en un mismo gr\'afico $f$ y la $N$-\'esima Suma Parcial de Fourier en el intervalo $[a,b]$.
\item Pruebe su funci\'on para distintos valores $N=1,2,5,10$ y las siguientes funciones:
\begin{enumerate}
\begin{multicols}{2}
  \item $f(x)=1$, $x\in[0,1]$.
  \item $g(x)=x$, $x\in[-1,2]$.
\end{multicols}
  \item $\displaystyle h(x)=\left\{
  \begin{array}{cc}
    1 & \textrm{si } x\in[-2,0[ \\
     &  \\
    x^2 & \textrm{si } x\in[0,1]
  \end{array}
  \right.$
\end{enumerate} 
Observe que ocurre a medida que crecen los valores de $N$.
\item Modifique la funci\'on anterior para  que calcule la Suma Parcial de Fourier en el intervalo $[a,b]$, pero que grafique dicha suma en el intervalo $[a-L,b+L]$, con $L=b-a$. >Qu\'e se puede observar?
\end{enumerate}
\textbf{Sugerencia:} Es muy \'util el uso de \verb"@(x)" en el c\'alculo de las integrales, considando que se integra una funci\'on que depende de $a$, $b,$ $n$ y la variable $x$.

\item Los pares $(x,y)$ que satisfacen
$$\frac{x^2}{a^2}+\frac{y^2}{b^2}=1$$
pertenecen a la elipse con centro en el origen de coordenas, semieje mayor de longitud $a$ y semieje menor de longitud $b$. La ecuaci\'on param\'etrica de esta elipse es:
$$(x(t), y(t)) = (a \cos(t), b \sen(t)),\quad t \in [-\pi,\pi]$$
Escriba un rutero \matlab\, en el que:
\begin{enumerate}
\item Grafique 200 puntos sobre la elipse de ecuaci\'on param\'etrica $(3 \cos(t), 5 \sen(t))$,  $t \in [-\pi, \pi]$.
\item Encuentre, con ayuda del comando \verb+quad+ o del comando \verb+quadl+ una aproximaci\'on al
per\'imetro de la elipse antes graficada. Tenga en cuenta que \'este es igual a
$$\int_{-\pi}^{\pi}\sqrt{\left(\frac{\textrm{d}x}{\textrm{d}t}\right)^2+\left(\frac{\textrm{d}y}{\textrm{d}t}\right)^2}\,\mathrm{d}t.$$
\end{enumerate}
\end{enumerate}

\vfill
\hfill Revisado a Semestre 2018--1
\end{document}
