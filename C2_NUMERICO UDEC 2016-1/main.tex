\documentclass[11pt]{article}

\usepackage{multicol}
\usepackage{graphicx}
\usepackage[english]{babel}
\usepackage[utf8]{inputenc}
\usepackage{amsmath}
\usepackage{amssymb}
\usepackage{listings}
\usepackage[numbered,framed]{matlab-prettifier}
\lstMakeShortInline"
\lstset{
  style              = Matlab-editor,
  %basicstyle         = \mlttfamily,
  escapechar         = ",
  mlshowsectionrules = true,
}

\newcommand{\buena}{[BUENA]}
\newcommand{\justifique}{[JUSTIFIQUE]}
\newcommand{\nojustifique}{[NOJUSTIFIQUE]}

\begin{document}
\title{C2_NUMERICO UDEC 2016-1}
\begin{itemize}

\item Determine cuales de las siguientes instrucciones de Matlab estima $\int_0^2 e^{ln(2x^2+1)} dx$ utilizando una regla del punto medio compuesta.
\begin{enumerate} 
\item
\begin{lstlisting}
x=0:0.1:2;
int=0.1*sum(exp(log(2*x.^2+1)));
\end{lstlisting}

\item
\begin{lstlisting}
x=0:0.05:2;
int=sum(0.1*exp(log(2*x.^2+1)));
\end{lstlisting}

\item \buena
\begin{lstlisting}
x=[0:0.1:1.9]+0.05;
int=0.1*sum(exp(log(2*x.^2+1)));
\end{lstlisting}

\item
\begin{lstlisting}
x=0:0.1:2;
int=0.1*sum(2*x.^2+1);
\end{lstlisting}
\end{enumerate}
\nojustifique 

\item Sobre el m\'etodo de la bisecci\'on se establecen las siguientes proposiciones
\begin{enumerate}
	\item Permite resolver algunas ecuaciones no lineales.
    \item Permite resolver algunos sistemas de ecuaciones no lineales.
    \item No permite resolver una ecuaci\'on lineal.
\end{enumerate}
Decida c\'ual de las siguientes opciones es correcta.
\begin{enumerate}
	\item Sólo i) y ii) son falsas.
    \item Sólo i) y iii) son falsas.
    \item \buena Sólo ii) y iii) son falsas.
    \item Sólo iii) es falsa.
\end{enumerate}
\nojustifique 

\item Sobre los m\'etodos num\'ericos que resuelven P.V.I. de orden 1, vistos en clases, se establecen las siguientes proposiciones
\begin{enumerate}
	\item No se pueden aplicar P.V.I. de orden 4.
    \item No permiten resolver sistemas de E.D.O.
    \item Algunos de ellos no requieren de condiciones iniciales.
\end{enumerate}
Decida cual de las siguientes opciones es correcta.
\begin{enumerate}
	\item Sólo iii) es falsa.
    \item Sólo i) y ii) son falsas.
    \item \buena  i), ii) y iii) son falsas.
    \item Sólo iii) es falsa.
\end{enumerate}
\nojustifique 

\item El siguiente P.V.I. modela el movimiento de un sistema masa-resorte amortiguado
$$
\begin{array}{rl|}
mx''(t)+cx'(t)+kx(t) 	&=0 \\
x(0) 					&=x_0\\
x'(0) 					&=x_1\\
\hline
\end{array}\,,
$$
donde $x(t)$ es la oscilaci\'on del sistema masa resorte respecto a su posici\'on de equilibrio. Sobre este sistema se establecen las siguientes proposiciones.
\begin{enumerate}
	\item Se puede resolver usando el m\'etodo de Newton.
    \item Se transforma en un sistema de E.D.O. con tres inc\'ognitas
    \item Se puede resolver usando un m\'etodos de Runge-Kutta.
\end{enumerate}
Decida cual de las siguientes opciones es correcta.
\begin{enumerate}
	\item Sólo i) es falsa.
    \item S\'olo ii) es falsa.
    \item \buena S\'olo iii) es verdadera.
    \item S\'olo ii) y iii) son verdaderas.
\end{enumerate}
\nojustifique 

\item Decida cu\'al de las siguientes proposiciones es falsa y justifique su elecci\'on mostrando un contraejemplo.
\begin{enumerate}
	\item Existen m\'etodos num\'ericos para resolver P.V.I. de orden 5.
    \item \buena Toda soluci\'on num\'erica de un P.V.I. generada por un m\'etodo Runge-Kutta tiene error global mayor que cero.
    \item Existen integrales que se no se pueden calcular exactamente con la regla del punto medio.
    \item El método de Newton permite encontrar ra\'ices de un sistema de ecuaciones no lineales.
\end{enumerate}
\justifique 
\end{itemize}
\end{document}