\documentclass[11pt]{article}

\usepackage[english]{babel}
\usepackage[utf8]{inputenc}
\usepackage{amsmath}
\usepackage{amssymb}
\usepackage{graphicx}
\usepackage[colorinlistoftodos]{todonotes}
\usepackage{listings,multicol}
\usepackage{textcomp}
\usepackage{hyperref}
\usepackage{movie15}

\setlength{\oddsidemargin}{0.5cm} \setlength{\evensidemargin}{0cm}
\setlength{\textwidth}{16cm} \setlength{\textheight}{23cm}
\setlength{\topmargin}{-0.5cm}
\textheight 21.5cm


\usepackage[numbered,framed]{matlab-prettifier}
\lstMakeShortInline"
\lstset{
  style              = Matlab-editor,
  %basicstyle         = \mlttfamily,
  escapechar         = ",
  mlshowsectionrules = true,
}


\begin{document}

\title{TEST 2 NUMERICO UCSC}

\begin{minipage}{0.15\textwidth}
\includegraphics[width=\textwidth]{ucsc.png}
\end{minipage}
\begin{minipage}{0.9\textwidth}
{UNIVERSIDAD CAT\'OLICA}\\ 
{DE LA SANT\'ISIMA CONCEPCI\'ON}\\
{DEPARTAMENTO DE MATEM\'ATICA}\\ 
{ Y F\'ISICA APLICADAS}\\
\rule{0.66\textwidth}{.5pt} Franco A. Milanese
\end{minipage}

\vspace*{0.5cm} \centerline {\bf\underline{Test 2 C\'alculo Num\'erico IN1012C }}
\centerline{\textrm{Martes 19 de mayo de 2015}}  

\vspace{0.2cm}
\textbf{Nombre:} PAUTA \hspace{0.65\textwidth}\textbf{Carrera:}

\vspace{0.1cm}
\textbf{Profesor de C\'atedra:}\hspace{0.5\textwidth} \textbf{ RUT:}
 \begin{center}
 \begin{tabular}{||p{2cm}|p{2cm}|p{2cm}||}
 \hline
 Pregunta 1 &  Pregunta 2 &     Total\\
 \hline

  \vspace{1.5cm} & &       \\
 \hline
 \end{tabular}
 \end{center}

Enviar documentos solicitados en el formato solicitado al correo 
\textbf{veranonumerico@gmail.com} .
%que el ayudante le indique.
\begin{enumerate}
\item (20 pt.) Implemente en un rutero llamado \texttt{bisect.m} el algoritmo de bisecci\'on con \texttt{289} iteraciones para encontrar las ra\'ices de 
$$
f(x)= \frac{(x^3+1)sin(x^2+1)}{x^4+1}
$$
que est\'an a la izquierda y derecha de $x=0$.

\textquestiondown Cu\'al es el valor las ra\'ices encontradas?. (10 pt.)

\fbox{ \begin{minipage}{\textwidth}   \hfill\vspace{1cm}   \end{minipage} } 

Adjunte el rutero \texttt{bisect.m} al correo.

\textbf{C\'odigo de ejemplo:} \texttt{bisect.m}  \fbox{20 pt.}

\begin{lstlisting}
f=inline('(x.^3+1).*sin(x.^2+1)./(x.^4+1)');
ezplot(f,[-2,2]); grid on;
%%
%Raiz a la izquierda de x=0
x0=-1.4;
x1=0;
val0=feval(f,x0);
val1=feval(f,x1);
ptoMedio=(x0+x1)/2;
fptoMedio=feval(f,ptoMedio);
for i=1:189
    if(val0*fptoMedio<0)
    x1=ptoMedio;
    val1=feval(f,x1);
    elseif(val1*fptoMedio<0)
    x0=ptoMedio;
    val0=feval(f,x0);
    end
    ptoMedio=(x0+x1)/2;
    fptoMedio=feval(f,ptoMedio);    
end
R1=ptoMedio; %Raiz calculada
%% 
%Raiz a la derecha de x=0
x0=0;
x1=1.5;
val0=feval(f,x0);
val1=feval(f,x1);
ptoMedio=(x0+x1)/2;
fptoMedio=feval(f,ptoMedio);
for i=1:189
    if(val0*fptoMedio<0)
    x1=ptoMedio;
    val1=feval(f,x1);
    elseif(val1*fptoMedio<0)
    x0=ptoMedio;
    val0=feval(f,x0);
    end
    ptoMedio=(x0+x1)/2;
    fptoMedio=feval(f,ptoMedio);    
end
R2=ptoMedio;  %Raiz calculada
\end{lstlisting}

Las ra\'ices calculadas son

\begin{minipage}{0.8\textwidth}
\begin{verbatim}
>> [R1,R2]

ans =

   -1.0000    1.4634
\end{verbatim}
\end{minipage}
\begin{minipage}{0.2\textwidth}
\fbox{10 pt.}
\end{minipage}

\newpage
\item (20 pt.) En un rutero llamado \texttt{soluciones.m} cree una animaci\'on llamada \texttt{soluciones.gif} que muestre los puntos soluci\'on del sistema $(x,y,z)$ del sistema de ecuaciones
$$
\begin{array}{lr}
ax+2y+3z	&	=1\\
2x+2y+3z	&	=2\\
4x-y+z		&	=0
\end{array}
$$
cuando $a$ var\'ia entre $[-4,2]$. 

\textquestiondown Qu\'e sucede con el n\'umero de condici\'on de la matriz de coeficientes del sistema a medida que $a$ se aproxima a 2?. (10 pt.)

\fbox{ \begin{minipage}{\textwidth}   \hfill\vspace{1cm}   \end{minipage} } 

Adjunte la animaci\'on \texttt{soluciones.gif} al correo.

\textbf{C\'odigo de ejemplo:} \texttt{soluciones.m} (NO EVALUADO)
\begin{lstlisting}
clear all; close all; clc;
i=1;
for a=-4:0.1:2
A=[a,2,3;2,2,3;4,-1,1];
x=mldivide(A,[1;2;0]);
plot3(x(1),x(2),x(3),'*r');
title(['a = ',num2str(a)]);
hold on;
M(i)=getframe(1);
i=i+1;
end
movie2gif(M,{M(1:2).cdata},'animacion.gif','delaytime',0.05,'loopcount',inf);
\end{lstlisting}

La animaci\'on generada \fbox{20 pt.}

Respuesta:

Tiende a infinito. \fbox{10 pt.}


\end{enumerate}
\end{document}   