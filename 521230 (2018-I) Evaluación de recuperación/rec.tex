\documentclass[letterpaper,12pt]{article}
\usepackage[spanish]{babel}
\usepackage[utf8]{inputenc}
\usepackage{amsthm}
\usepackage{amsmath}
\usepackage{amssymb}
\usepackage{tabularx}
\usepackage{verbatim} % Para el ambiente comment
\usepackage{dashundergaps} % Para \dashuline
\usepackage{srcltx}

\theoremstyle{definition}
\newtheorem{question}{Pregunta}
\renewcommand{\thequestion}{\Alph{question}}
\numberwithin{equation}{question}

\newenvironment{solution}{\begin{proof}[Solución]}{\end{proof}}
%\newenvironment{solution}{\comment}{\endcomment}

\renewcommand{\theenumi}{\roman{enumi}}

\pagestyle{empty}
\usepackage[left=1.5cm, right=1.5cm, top=1.5cm, bottom=2cm]{geometry}

\title{521230 (2018-I): Evaluación de recuperación} %For overleaf only

\begin{document}
\font\bff=cmbx10 at 10truept
\font\lg=cmdunh10 at 10truept
\font\bl=cmss10 at 10truept

\noindent{\lg UNIVERSIDAD DE CONCEPCIÓN}\hfill
\vskip-2truept
\noindent{\bff FACULTAD DE CIENCIAS FÍSICAS Y MATEMÁTICAS}\hfill
\vskip-2truept
\noindent{\bl DEPARTAMENTO DE INGENIERÍA MATEMÁTICA}\hfill
\vskip4truept\hrule\hrule\vskip4truept
\par
\bigskip
\begin{center}
\large{\textbf{521230 Cálculo Numérico (2018-I)}}\\
\large{\textbf{Evaluación de Recuperaci\'on}
\par \medskip 25 de julio de 2018}
\end{center}

\bigskip

\noindent%
\textbf{Nombre:} \dashuline{\hfill}\\[1.5ex]
\textbf{Número de matrícula:} \dashuline{\hfill}\\[1.5ex]
\textbf{Sección:} {\small $\square$ 1 (Prof.~Leonardo Figueroa C.) \hfill $\square$ 2 (Prof.\ Franco Milanese P.) \hfill $\square$ 3 (Prof.~Mauricio Vega H.)}

\bigskip

Esta evaluación consta de \textbf{tres} preguntas con la misma ponderación.
No se permite el uso de calculadoras u otros dispositivos electrónicos.
Duración: 100 minutos.

\begin{question}[20 puntos]
Según lo visto en clase, usando la notación habitual, en el método de Adams--Moulton de orden 2,
%
\begin{equation}\label{general-multistep}
y_{i+1} = y_i + \int_{x_i}^{x_{i+1}} p(x) \, \mathrm{d} x,
\end{equation}
%
donde $p$ es el único polinomio de grado menor o igual a $k = 1$ que interpola los pares ordenados $(x_i, f_i)$ y $(x_{i+1}, f_{i+1})$.
\textbf{Calcule} a partir de lo anterior los números $\sigma_i$ y $\sigma_{i+1}$, que deben ser independientes del tamaño de paso $h$, tales que
%
\begin{equation*}
y_{i+1} = y_i + h \left(\sigma_i f_i + \sigma_{i+1} f_{i+1}\right)
\end{equation*}
%
sea el método de Adams--Moulton de orden 2.
\textbf{Nota:} Esta no es una pregunta de memoria.
\begin{solution}

Usando interpolación de Lagrange, el polinomio $p$ es
%
\begin{equation*}
p(x) = \frac{x - x_{i+1}}{x_i - x_{i+1}} f_i + \frac{x - x_i}{x_{i+1} - x_i} f_{i+1}
= -\frac{x - x_{i+1}}{h} f_i + \frac{x-x_i}{h} f_{i+1}.
\end{equation*}
%
Así, sustituyendo en \eqref{general-multistep},
%
\begin{multline*}
y_{i+1} = y_i + \int_{x_i}^{x_{i+1}} \left[ -\frac{x - x_{i+1}}{h} f_i + \frac{x-x_i}{h} f_{i+1} \right] \, \mathrm{d} x
= y_i + \left.\left[ -\frac{(x - x_{i+1})^2}{2h} f_i + \frac{(x-x_i)^2}{2h} f_{i+1} \right]\right|_{x=x_i}^{x=x_{i+1}}\\
= y_i + \left[ -0 + \frac{(x_{i+1} - x_i)^2}{2h} f_{i+1} + \frac{(x_i - x_{i+1})^2}{2h} f_i - 0 \right]
= y_i + \left[ \frac{h^2}{2h} f_{i+1} + \frac{(-h)^2}{2h} f_i \right]\\
= y_i + h \left( \frac{1}{2} f_i + \frac{1}{2} f_{i+1} \right),
\end{multline*}
%
de donde $\sigma_i = \sigma_{i+1} = \frac{1}{2}$.

\bigskip
\begin{center}\framebox{\begin{minipage}{0.9\linewidth}
Definir el polinomio de interpolaci\'on apropiado: 8 puntos.
Desarrollar la integral hasta obtener la f\'ormula del m\'etodo de A-M: 10 puntos.
Identificar los par\'ametros $\sigma$: 2 puntos.
\end{minipage}}\end{center}

\end{solution}
\end{question}

\newpage

\begin{question}[20 puntos]
Construya una regla de cuadratura de la forma
%
\begin{equation}\label{weirdRule}
\int_0^1 f(\sqrt{x})\,\mathrm{d}x \approx w_1 \, f(0) + w_2 \, f(1/2) + w_3 \, f(1)
\end{equation}
%
que sea exacta para todos los polinomios de grado menor o igual a dos.
Aplique esta f\'ormula con los $w_1$, $w_2$ y $w_3$ obtenidos para estimar la integral
%
\begin{equation}
I = \int_0^1\sqrt{1+\sqrt{x}}\,\mathrm{d}x
\end{equation}
%
y luego calcule el error cometido por esta cuadratura (use que el valor exacto de $I$ es $\frac{8(\sqrt{2}+1)}{15}$).
\end{question}

\begin{solution}
Para que la regla de cuadratrua \eqref{weirdRule} sea exacta para polinomios de grado dos, debe ser exacta en cualquier base de este espacio. En particular, usando la base can\'onica de monomios,
\begin{align*}
f(t)=1 		&\quad\longrightarrow\quad 1=\int_0^1 1 \, \mathrm{d}x = w_1+w_2+w_3,\\
f(t)=t 		&\quad\longrightarrow\quad \frac{2}{3}=\int_0^1 \sqrt{x} \, \mathrm{d}x = 0w_1+\frac{1}{2}w_2+w_3,\\
f(t)=t^2	&\quad\longrightarrow\quad \frac{1}{2}=\int_0^1 \left(\sqrt{x}\right)^2 \, \mathrm{d}x = 0w_1+\frac{1}{4}w_2+w_3,
\end{align*}
de donde se llega al sistema
$$
\begin{bmatrix}
1 & 1 & 1 \\
0 & 1/2 & 1\\
0 & 1/4 & 1 
\end{bmatrix}
\begin{bmatrix}
w_1\\w_2\\w_3
\end{bmatrix}
=
\begin{bmatrix}
1\\ \frac{2}{3}\\1
\end{bmatrix}
\rightarrow
w_1 = 0,\quad
w_2 = \frac{2}{3},\quad
w_3 = \frac{1}{3}.
$$
De donde, aplicando la regla de cuadratura \eqref{weirdRule} a la función $f(t) = \sqrt{1+t}$,
\begin{equation*}
\displaystyle \int_0^1\sqrt{1+\sqrt{x}} \, \mathrm{d}x \approx
0 \, \sqrt{1+0} + \frac{2}{3} \, \sqrt{1+\frac{1}{2}} + \frac{1}{3} \, \sqrt{1+1}
= \frac{2}{3} \sqrt{\frac{3}{2}} + \frac{1}{3} \sqrt{2}
= \frac{\sqrt{2}}{3} (\sqrt{3} + 1).
\end{equation*}
As\'i, el error de esta regla aplicada a este caso es
$$
\frac{8(\sqrt{2}+1)}{15} - \frac{\sqrt{2}}{3} (\sqrt{3} + 1).
$$
\bigskip
\begin{center}\framebox{\begin{minipage}{0.9\linewidth}
Construir el sistema de ecuaciones utilizando los polinomios apropiados: 5 puntos.
Resolver el sistema para obtener los par\'ametros $w$: 5 puntos.
Calcular la integral usando la f\'ormula obtenida: 5 puntos. 
Calcular el error: 5 puntos.
\end{minipage}}\end{center}

\end{solution}


\newpage
\begin{question}[20 puntos]
Los datos de la siguiente tabla, muestran los resultados de un experimento donde la variable $y$ depende del tiempo $t$:
\begin{center}
\begin{tabular}{c|cccc}
$t$ & 1 & 2 & 6 & 13\\
\hline
$y(t)$ & $-\frac{3}{2}$ & 1 & $\frac{3}{4}$ & $\frac{1}{2}$
\end{tabular}
\end{center}
Se desea ajustar estos datos al modelo
%
\begin{equation}\label{modelo}
y(t)=\dfrac{t}{a \, 3^{bt}-1}.
\end{equation}
%
Para ello:
\begin{enumerate}
\setlength{\itemsep}{-1ex}
\item\label{it:transform} aplique una transformación apropiada,
\item\label{it:system} escriba el sistema rectangular que habría que resolver en el sentido de los mínimos cuadrados,
\item\label{it:solve} resuelva el sistema obtenido
\item\label{it:recover-parameters} y obtenga los parámetros $a$ y $b$ que se desean.
\end{enumerate}
\begin{solution}
A partir de \eqref{modelo}, tomando recíprocos, multiplicando por $t$, sumando $1$ y tomando logaritmo natural \footnote{Tambi\'en es v\'alido tomar el logaritmo en base 3, lo cual simplifica los c\'alculos.} a ambos lados obtenemos
%
\begin{equation*}
\ln(a) + \ln(3) \, t \, b = \ln\left(\frac{t}{y(t)} + 1\right).
\end{equation*}
%
Sustituyendo los datos de la tabla obtenemos el sistema rectangular de ecuaciones lineales
%
\begin{equation*}
\begin{pmatrix}
1 & \ln(3)\\
1 & 2\ln(3)\\
1 & 6\ln(3)\\
1 & 13\ln(3)
\end{pmatrix}
\begin{pmatrix}
\ln(a)\\ b
\end{pmatrix}
= \begin{pmatrix}
-\ln(3)\\ \ln(3) \\ 2 \ln(3) \\ 3\ln(3)
\end{pmatrix}.
\end{equation*}
%
El sistema normal asociado es
%
\begin{equation*}
\begin{pmatrix}
4 & 22 \ln(3)\\ 22 \ln(3) & 210 \ln(3)^2
\end{pmatrix}
\begin{pmatrix}
\ln(a)\\ b
\end{pmatrix}
= \begin{pmatrix}
5 \ln(3)\\ 52 \ln(3)^2
\end{pmatrix},
\end{equation*}
%
cuya solución es
%
\begin{equation*}
\ln(a) = -\frac{47}{178}\ln(3), \quad b = \frac{49}{178}.
\end{equation*}
%
Para obtener $a$ tomamos la exponencial.
Así, finalmente,
%
\begin{equation*}
a = 3^{-\frac{47}{178}}, \quad b = \frac{49}{178}.
\end{equation*}
\bigskip
\begin{center}\framebox{\begin{minipage}{0.9\linewidth}
Reducir \eqref{modelo} a un modelo lineal: 7 puntos.
Obtener el sistema rectangular: 3 puntos.
Resolver el sistema rectangular en el sentido de los m\'inimos cuadrados: 7 puntos.
Obtener los par\'ametros $a$ y $b$: 3 puntos ( 2 y 1 respectivamente).
\end{minipage}}\end{center}
\end{solution}
\end{question}

\end{document}
