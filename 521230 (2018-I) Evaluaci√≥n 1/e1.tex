\documentclass[letter,12pt]{article}
\usepackage[spanish]{babel}
\usepackage[utf8]{inputenc}
\usepackage{amsthm}
\usepackage{amsmath}
\usepackage{amssymb}
\usepackage{tabularx}
\usepackage{verbatim} % Para el ambiente comment
\usepackage{dashundergaps} % Para \dashuline
\usepackage{srcltx}

\theoremstyle{definition}
\newtheorem{question}{Pregunta}
\renewcommand{\thequestion}{\Alph{question}}
\numberwithin{equation}{question}

\newenvironment{solution}{\begin{proof}[Solución]}{\end{proof}}

\renewcommand{\theenumi}{\roman{enumi}}

\pagestyle{empty}
\usepackage[left=1.5cm, right=1.5cm, top=1.5cm, bottom=2cm]{geometry}

\begin{document}
\font\bff=cmbx10 at 10truept
\font\lg=cmdunh10 at 10truept
\font\bl=cmss10 at 10truept

\noindent{\lg UNIVERSIDAD DE CONCEPCIÓN}\hfill
\vskip-2truept
\noindent{\bff FACULTAD DE CIENCIAS FÍSICAS Y MATEMÁTICAS}\hfill
\vskip-2truept
\noindent{\bl DEPARTAMENTO DE INGENIERÍA MATEMÁTICA}\hfill
\vskip4truept\hrule\hrule\vskip4truept
\par
\bigskip
\begin{center}
\large{\textbf{521230 Cálculo Numérico (2018-I)}}\\
\large{\textbf{Evaluación 1}
\par \medskip 16 de mayo de 2018}
\end{center}

\bigskip

\noindent%
\textbf{Nombre:} \dashuline{\hfill}\\[1.5ex]
\textbf{Número de matrícula:} \dashuline{\hfill}\\[1.5ex]
\textbf{Sección:} {\small $\square$ 1 (Prof.~Leonardo Figueroa C.) \hfill $\square$ 2 (Prof.\ Franco Milanese P.) \hfill $\square$ 3 (Prof.~Mauricio Vega H.)}

\bigskip

Esta evaluación consta de \textbf{cuatro} preguntas con la misma ponderación.
No se permite el uso de calculadoras u otros dispositivos electrónicos.
Duración: 100 minutos.

\begin{question}[\framebox{15 puntos}] Se desea aproximar una solución de la ecuación
%
\begin{equation*}
x \, \exp(x) = 2.
\end{equation*}
%
\begin{enumerate}
\item\label{it:bisec} (50\%) Efectúe una iteración del método de la bisección aplicado a este problema inicializando con $a = -1$ y $b = 1$.
\item\label{it:Newton} (50\%) Efectúe una iteración del método de Newton aplicado a este problema inicializando con $x_0 = 0$.
\end{enumerate}

\begin{solution}
\textbf{Parte \ref{it:bisec}.}

\framebox{1.5 puntos} Definiendo la función $f$ por $f(x) = x\,\exp(x)-2$, el problema se reduce a buscar una raíz de $f$.

\framebox{5 puntos} Como $f$ es continua en $[a,b]$ y $f(a) = -\exp(-1)-2 < 0$ y $f(b) = \exp(1)-2 > 0$, podemos aplicar el método de la bisección.
El primer punto medio es $\hat x = 0$.
Como $f(0) = -2 < 0$, redefinimos $a = 0$.
Nuestro nuevo intervalo de trabajo es $[a,b] = [0,1]$.

\framebox{1 punto} La respuesta de la primera iteración es el punto medio del nuevo intervalo; esto es, $\hat x = \frac{1}{2}$.

\medskip

\textbf{Parte \ref{it:Newton}.}

\smallskip
\noindent\framebox{\begin{minipage}{\linewidth}
1.5 puntos --- Si se identificó una función apropiada como $f$ en la parte anterior, también se asigna el puntaje.
\end{minipage}}
Definiendo la función $f$ por $f(x) = x\,\exp(x)-2$, el problema se reduce a buscar una raíz de $f$.


\framebox{2 puntos} Calculamos $f'(x) = x \, \exp(x) + \exp(x) = (x+1) \exp(x)$.
Entonces, el resultado de aplicar una iteración del método de Newton a este problema es
%
\begin{equation*}
\text{\framebox{4 puntos}} \quad x_1 = x_0 - \frac{f(x_0)}{f'(x_0)} = 0 - \frac{(-2)}{1} = 2.
\end{equation*}

\end{solution}
\end{question}

\newpage
\begin{question}[\framebox{15 puntos}] Sea $f \colon [0,2] \to \mathbb{R}$ la función definida por
%
\begin{equation*}
(\forall\,x\in[0,2]) \quad f(x)
= \begin{cases} x-2 & \text{si } 0 \leq x \leq 1,\\
-2x + 1 & \text{si } 1 < x \leq 2.
\end{cases}
\end{equation*}
%
\begin{enumerate}
\item\label{it:coarse} (50\%) Calcule la aproximación de la integral $I = \int_0^2 f(x) \, \mathrm{d}x$ producida por el método del trapecio compuesto con tamaño de paso $h = 2/3$.
\item\label{it:fine} (50\%) Calcule la aproximación de la misma integral $I$ producida por el método del trapecio compuesto con tamaño de paso $h = 1/32$.
\end{enumerate}

\begin{solution}
\textbf{Parte \ref{it:coarse}.}

\smallskip
\noindent\framebox{\begin{minipage}{\textwidth}
Recordar fórmula o definición del método del trapecio compuesto: 2 puntos;
usar el $h$ indicado y no otro: 3 puntos;
realizar bien las operaciones aritméticas: 2.5 puntos
\end{minipage}}

Con tamaño de paso $h = 2/3$ la aproximación es
%
\begin{multline*}
\frac{2/3}{2}\left[f(0) + 2 f(2/3) + 2 f(4/3) + f(2)\right]
= \frac{1}{3}\left[(0-2) + 2(2/3-2) + 2(-2\times4/3+1) + (-2\times2+1)\right]\\
= \frac{1}{3}\left[-2 + -8/3 - 10/3 - 3\right]
= \frac{1}{3} \frac{(-33)}{3} = -\frac{11}{3}.
\end{multline*}

\medskip

\textbf{Parte \ref{it:fine}.}

\smallskip
\noindent\framebox{\begin{minipage}{\textwidth}
Recordar fórmula o definición del método del trapecio compuesto: 2 puntos;
usar el $h$ indicado y no otro: 3 puntos;
realizar bien las operaciones aritméticas incluyendo, de ser necesario, evaluación de sumas: 2.5 puntos
\end{minipage}}

Al aplicar el método del trapecio compuesto con paso $h = 1/32$, lo que estamos haciendo, por definición, es descomponer
%
\begin{equation}\label{aux}
I = \sum_{i=1}^{n} \int_{x_{i-1}}^{x_i} f(x) \, \mathrm{d}x,
\end{equation}
%
y aplicar el método del trapecio elemental a cada integral dentro de la suma en \eqref{aux}.
En \eqref{aux} $n$ está relacionado con $h$ a través de $h = (2-0)/n$ (esto es, $n = 2/h = 64$) y, para todo $i \in \{0, \dotsc, n\}$, $x_i = 0 + i\,h = i/32$.
Ahora, si $i \leq 32$, $x_i \leq 1$.
Por otro lado, si $i \geq 33$, $x_{i-1} = (i-1)/32 \geq (33-1)/32 = 1$.
Entonces, por la definición de $f$,
%
\begin{subequations}\label{eux}
\begin{gather}
(\forall\,i\in\{1,\dotsc,32\}) \quad \int_{x_{i-1}}^{x_i} f(x) \,\mathrm{d}x = \int_{x_{i-1}}^{x_i} (x-2) \,\mathrm{d}x,\\
(\forall\,i\in\{33,\dotsc,64\}) \quad \int_{x_{i-1}}^{x_i} f(x) \,\mathrm{d}x = \int_{x_{i-1}}^{x_i} (-2x+1) \,\mathrm{d}x.
\end{gather}
\end{subequations}
%
Como tanto $x \mapsto x-2$ y $x \mapsto -2x+1$ son polinomios de grado menor o igual a $1$, la regla del trapecio elemental aproxima a estas integrales en \eqref{eux} en forma exacta.
Por lo tanto, la aproximación de $I$ es
%
\begin{multline*}
\sum_{i=1}^{32} \int_{x_{i-1}}^{x_i} (x-2) \,\mathrm{d}x + \sum_{i=33}^{64} \int_{x_{i-1}}^{x_i} (-2x+1) \, \mathrm{d}x
= \int_0^1 (x-2) \, \mathrm{d}x + \int_1^2 (-2x+1) \, \mathrm{d} x\\
= \left.\left(\frac{1}{2} x^2 - 2x \right)\right|_{x=0}^{x=1} + \left.\left(-x^2 + x\right)\right|_{x=1}^{x=2}
= \frac{1}{2} - 2 -4 + 2 - (-1) - 1
= -\frac{7}{2}.
\end{multline*}
%
\textbf{Alternativamente}, la aproximación de $I$ que hace el método del trapecio compuesto con paso $h = 1/32$ se puede calcular utilizando la fórmula derivada en clase:
%
\begin{multline*}
\frac{1/32}{2} \left(f(0) + 2 \sum_{i=1}^{63} f(x_i) + f(2)\right)
= \frac{1}{64} \left(f(0) + 2 \sum_{i=1}^{32} f(i/32) + 2 \sum_{i=33}^{63} f(i/32) + f(2)\right)\\
= \frac{1}{64} \left(-2 + 2 \sum_{i=1}^{32} (i/32-2) + 2 \sum_{i=33}^{63} (-2\,i/32 + 1) -3\right)\\
= \frac{1}{64} \left(-2 + \sum_{i=1}^{32} (i/16-4) + \sum_{i=33}^{63} (-i/8 + 2) -3\right)\\
= \frac{1}{64} \left(-2 - 4\times 32 + 2\times(63-33+1) - 3 + \frac{1}{16} \sum_{i=1}^{32} i - \frac{1}{8} \left[\sum_{i=1}^{63} i - \sum_{i=1}^{32} i\right]\right)\\
= \frac{1}{64} \left(-71 + \frac{32(32+1)}{16 \times 2} - \frac{63(63+1)}{8 \times 2} + \frac{32(32+1)}{8\times2} \right)
= \frac{1}{64}\left(-71 + 33 - 63\times 4 + 2 \times 33\right) = -\frac{7}{2}.
\end{multline*}
%
\end{solution}
\end{question}

\newpage
\begin{question}[\framebox{15 puntos}]\hfill
\begin{enumerate}
\item\label{it:fit} (80\%) Ajuste por mínimos cuadrados los coeficientes $\alpha_1$, $\alpha_2$ y $\alpha_3$ del modelo
%
\begin{equation*}
f(x) = \alpha_1 + \alpha_2 x + \alpha_3 \left(\frac{x^2}{2} - 1\right)
\end{equation*}
%
a los datos de la tabla
%
\begin{center}
\begin{tabular}{c|cccc}
$x_i$ & $-\sqrt{2+\sqrt{2}}$ & $-\sqrt{2-\sqrt{2}}$ & $\sqrt{2-\sqrt{2}}$ & $\sqrt{2+\sqrt{2}}$\\\hline
$f(x_i)$ & -4 & 0 & 0 & 4
\end{tabular}
\end{center}
\item\label{it:assess} (20\%) ¿El modelo ajustado interpola los datos?
\end{enumerate}

\begin{solution}
\textbf{Parte \ref{it:fit}.}

\framebox{Planteo del sistema rectangular: 4 puntos} Las ecuaciones que deseamos ajustar son
%
\begin{align*}
\alpha_1 - \alpha_2 \sqrt{2+\sqrt{2}} + \alpha_3 \left(\left(-\sqrt{2+\sqrt{2}}\right)^2/2-1\right)  & = -4,\\
\alpha_1 - \alpha_2 \sqrt{2-\sqrt{2}} + \alpha_3 \left(\left(-\sqrt{2-\sqrt{2}}\right)^2/2-1\right)  & = 0,\\
\alpha_1 + \alpha_2 \sqrt{2-\sqrt{2}} + \alpha_3 \left(\left(\sqrt{2-\sqrt{2}}\right)^2/2-1\right)  & = 0,\\
\alpha_1 + \alpha_2 \sqrt{2+\sqrt{2}} + \alpha_3 \left(\left(\sqrt{2+\sqrt{2}}\right)^2/2-1\right)  & = 4.
\end{align*}
%
En términos de matrices y vectores, podemos escribir esto como que deseamos ajustar el sistema no-rectangular de ecuaciones $\boldsymbol{A} \boldsymbol{\alpha} = \boldsymbol{b}$, donde $\boldsymbol{\alpha} = \begin{pmatrix} \alpha_1 & \alpha_2 & \alpha_3 \end{pmatrix}^{\mathrm{T}}$, $\boldsymbol{b} = \begin{pmatrix} -4 & 0 & 0 & 4 \end{pmatrix}^{\mathrm{T}}$ y, usando que
%
\begin{gather*}
\left(-\sqrt{2+\sqrt{2}}\right)^2/2-1 = \left(\sqrt{2+\sqrt{2}}\right)^2/2-1 = \frac{2+\sqrt{2}}{2} - 1 = \frac{\sqrt{2}}{2}
\intertext{y}
\left(-\sqrt{2-\sqrt{2}}\right)^2/2-1 = \left(\sqrt{2-\sqrt{2}}\right)^2/2-1 = \frac{2-\sqrt{2}}{2} - 1 = -\frac{\sqrt{2}}{2},
\end{gather*}
%
la matriz $\boldsymbol{A}$ está dada por
%
\begin{equation*}
\boldsymbol{A} = \begin{pmatrix}
1 & -\sqrt{2+\sqrt{2}} & \frac{\sqrt{2}}{2}\\
1 & -\sqrt{2-\sqrt{2}} & -\frac{\sqrt{2}}{2}\\
1 & \sqrt{2-\sqrt{2}} & -\frac{\sqrt{2}}{2}\\
1 & \sqrt{2+\sqrt{2}} & \frac{\sqrt{2}}{2}
\end{pmatrix}.
\end{equation*}
%
\framebox{Planteo del sistema normal: 4 puntos}
La solución al sistema normal de ecuaciones $\boldsymbol{A}^{\mathrm{T}} \boldsymbol{A} \boldsymbol{\alpha} = \boldsymbol{A}^{\mathrm{T}} \boldsymbol{b}$ es la solución en el sentido de mínimos cuadrados del sistema no-rectangular $\boldsymbol{A} \boldsymbol{\alpha} = \boldsymbol{b}$.
Calculamos
%
\begin{equation*}
\boldsymbol{A}^{\mathrm{T}} \boldsymbol{A} =
\begin{pmatrix}
1 & 1 & 1 & 1\\
-\sqrt{2+\sqrt{2}} & -\sqrt{2-\sqrt{2}} & \sqrt{2-\sqrt{2}} & \sqrt{2+\sqrt{2}}\\
\frac{\sqrt{2}}{2} & -\frac{\sqrt{2}}{2} & -\frac{\sqrt{2}}{2} & \frac{\sqrt{2}}{2}
\end{pmatrix}
\begin{pmatrix}
1 & -\sqrt{2+\sqrt{2}} & \frac{\sqrt{2}}{2}\\
1 & -\sqrt{2-\sqrt{2}} & -\frac{\sqrt{2}}{2}\\
1 & \sqrt{2-\sqrt{2}} & -\frac{\sqrt{2}}{2}\\
1 & \sqrt{2+\sqrt{2}} & \frac{\sqrt{2}}{2}
\end{pmatrix}
= \begin{pmatrix}
4 & 0 & 0\\
0 & 8 & 0\\
0 & 0 & 2
\end{pmatrix}
\end{equation*}
%
y
\begin{equation*}
\boldsymbol{A}^{\mathrm{T}} \boldsymbol{b}
= \begin{pmatrix}
1 & 1 & 1 & 1\\
-\sqrt{2+\sqrt{2}} & -\sqrt{2-\sqrt{2}} & \sqrt{2-\sqrt{2}} & \sqrt{2+\sqrt{2}}\\
\frac{\sqrt{2}}{2} & -\frac{\sqrt{2}}{2} & -\frac{\sqrt{2}}{2} & \frac{\sqrt{2}}{2}
\end{pmatrix}
\begin{pmatrix} -4 \\ 0 \\ 0 \\ 4 \end{pmatrix}
= \begin{pmatrix} 0 \\ 8\sqrt{2+\sqrt{2}} \\ 0\end{pmatrix}.
\end{equation*}
%
\framebox{Obtención y uso de parámetros de ajuste: 4 puntos} Resolviendo, $\boldsymbol{\alpha} = \begin{pmatrix} 0 & \sqrt{2+\sqrt{2}} & 0 \end{pmatrix}^{\mathrm{T}}$.
El modelo ajustado es entonces $f(x) = \sqrt{2+\sqrt{2}} \, x$.

\medskip

\textbf{Parte \ref{it:assess}.}

\framebox{3 puntos} El modelo ajustado no interpola los datos, porque, por ejemplo,
%
\begin{equation*}
f\left(\sqrt{2-\sqrt{2}}\right) = \sqrt{2+\sqrt{2}} \sqrt{2-\sqrt{2}} \neq 0.
\end{equation*}
%
\end{solution}
\end{question}

\newpage
\begin{question}[\framebox{15 puntos}] El polinomio de Legendre de grado $3$ es $p_3(x) = \frac{1}{2}(5 x^3 - 3 x)$.
Calcule los nodos y los pesos de la regla de cuadratura de Gauss--Legendre de $3$ puntos.

\begin{solution} \framebox{6 puntos} Los nodos de la cuadratura de Gauss--Legendre de $3$ puntos son las raíces de $p_3$.
Claramente $0$ es una raíz de $p_3$.
Factorizando el monomio asociado a esa raíz, hallamos que $p_3(x) = \frac{5}{2} x \left(x^2 - \frac{3}{5}\right)$.
Desde esta última expresión obtenemos que las otras dos raíces son $-\sqrt{\frac{3}{5}}$ y $\sqrt{\frac{3}{5}}$.
En resumen, lo nodos son
%
\begin{equation*}
x_1 = -\sqrt{\frac{3}{5}}, \quad x_2 = 0 \quad\text{y}\quad x_3 = \sqrt{\frac{3}{5}}.
\end{equation*}
%

\smallskip
\noindent\framebox{\begin{minipage}{\linewidth}
Planteo de una estrategia apropiada para obtener pesos: 5 puntos; realizar bien las operaciones aritméticas: 4 puntos
\end{minipage}}
Sabemos que la cuadratura de Gauss--Legendre de $3$ puntos es exacta para polinomios de grado menor o igual a $2\times 3 - 1 = 5$.
En particular, es exacta para los polinomios $x \mapsto 1$, $x \mapsto x$ y $x \mapsto x^2$ (cualquier otra base del espacio de los polinomios de grado menor o igual que $2$ también sirve).
Por lo tanto, denotando por $w_1$, $w_2$ y $w_3$ a los pesos asociados a los nodos $x_1$, $x_2$ y $x_3$, respectivamente,
%
\begin{align*}
w_1 + w_2 + w_3 = \int_{-1}^1 1 \, \mathrm{d} x & = 2,\\
-\sqrt{\frac{3}{5}} \, w_1 + 0\,w_2 + \sqrt{\frac{3}{5}} \, w_3 = \int_{-1}^1 x \, \mathrm{d}x & = 0,\\
\frac{3}{5} \, w_1 + 0\,w_2 + \frac{3}{5} \, w_3 = \int_{-1}^1 x^2 \, \mathrm{d}x & = \frac{2}{3}.
\end{align*}
%
De la segunda ecuación se obtiene que $w_1 = w_3$ y de la tercera que $2 \times \frac{3}{5} w_1 = \frac{3}{5} w_1 + \frac{3}{5} w_3 = \frac{2}{3}$, de donde $w_1 = w_3 = \frac{5}{9}$.
De la primera ecuación, $\frac{5}{9} + w_2 + \frac{5}{9} = 2$, de donde $w_2 = \frac{8}{9}$.
En resumen, los pesos son
%
\begin{equation*}
w_1 = \frac{5}{9}, \quad w_2 = \frac{8}{9} \quad\text{y}\quad w_3 = \frac{5}{9}.
\end{equation*}
%
\textbf{Alternativamente}, los pesos se pueden calcular como las integrales de los polinomios de Lagrange asociados a los nodos:
%
\begin{multline*}
w_1 = \int_{-1}^1 \frac{x-0}{-\sqrt{3/5}-0} \, \frac{x-\sqrt{3/5}}{-\sqrt{3/5}-\sqrt{3/5}} \, \mathrm{d}x
= \frac{1}{2 \times 3/5} \int_{-1}^1 x \left(x-\sqrt{3/5}\right) \, \mathrm{d}x\\
= \frac{5}{6} \int_{-1}^1 x^2 \, \mathrm{d} x - \frac{5}{6}\sqrt{3/5} \underbrace{\int_{-1}^1 x \, \mathrm{d}x}_{= 0} = \frac{5}{6} \, \frac{2}{3} = \frac{5}{9},
\end{multline*}
%
\begin{multline*}
w_2 = \int_{-1}^1 \frac{x+\sqrt{3/5}}{0 + \sqrt{3/5}} \, \frac{x-\sqrt{3/5}}{0-\sqrt{3/5}} \, \mathrm{d}x
= -\frac{1}{3/5} \int_{-1}^1 \left(x+\sqrt{3/5}\right) \left(x-\sqrt{3/5}\right) \, \mathrm{d}x
= -\frac{5}{3} \int_{-1}^1 \left(x^2 - \frac{3}{5}\right) \, \mathrm{d}x\\
= -\frac{5}{3} \left( \frac{2}{3} - \frac{3}{5}\times 2\right)
= -\frac{5}{3} \left(-\frac{8}{15}\right) = \frac{8}{9},
\end{multline*}
%
\begin{multline*}
w_3 = \int_{-1}^1 \frac{x+\sqrt{3/5}}{\sqrt{3/5}+\sqrt{3/5}} \, \frac{x-0}{\sqrt{3/5}-0} \, \mathrm{d}x
= \frac{1}{2\times 3/5} \int_{-1}^1 \left(x+\sqrt{3/5}\right) x \, \mathrm{d}x\\
= \frac{5}{6} \int_{-1}^1 x^2 \, \mathrm{d}x + \frac{5}{6} \sqrt{3/5} \underbrace{\int_{-1}^1 x \, \mathrm{d}x}_{= 0} = \frac{5}{6} \, \frac{2}{3} = \frac{5}{9}.
\end{multline*}
\end{solution}
\end{question}


\end{document}
