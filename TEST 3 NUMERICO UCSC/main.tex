\documentclass[11pt]{article}

\usepackage[english]{babel}
\usepackage[utf8]{inputenc}
\usepackage{amsmath}
\usepackage{amssymb}
\usepackage{graphicx}
\usepackage[colorinlistoftodos]{todonotes}
\usepackage{listings,multicol}
\usepackage{textcomp}
\usepackage{hyperref}
\usepackage{movie15}

\setlength{\oddsidemargin}{0.5cm} \setlength{\evensidemargin}{0cm}
\setlength{\textwidth}{16cm} \setlength{\textheight}{23cm}
\setlength{\topmargin}{-0.5cm}
\textheight 21.5cm


\usepackage[numbered,framed]{matlab-prettifier}
\lstMakeShortInline"
\lstset{
  style              = Matlab-editor,
  %basicstyle         = \mlttfamily,
  escapechar         = ",
  mlshowsectionrules = true,
}


\begin{document}

\title{TEST 3 NUMERICO UCSC}

\begin{minipage}{0.15\textwidth}
\includegraphics[width=\textwidth]{ucsc.png}
\end{minipage}
\begin{minipage}{0.9\textwidth}
{UNIVERSIDAD CAT\'OLICA}\\ 
{DE LA SANT\'ISIMA CONCEPCI\'ON}\\
{DEPARTAMENTO DE MATEM\'ATICA}\\ 
{ Y F\'ISICA APLICADAS}\\
\rule{0.66\textwidth}{.5pt} Franco A. Milanese
\end{minipage}

\vspace*{0.5cm} \centerline {\bf\underline{Test 3 C\'alculo Num\'erico IN1012C }}
\centerline{\textrm{Lunes 2 de junio de 2015}}  

\vspace{0.2cm}
\textbf{Nombre:} PAUTA \hspace{0.65\textwidth}\textbf{Carrera:}

\vspace{0.1cm}
\textbf{Profesor de C\'atedra:}\hspace{0.5\textwidth} \textbf{ RUT:}
 \begin{center}
 \begin{tabular}{||p{2cm}|p{2cm}|p{2cm}||}
 \hline
 Pregunta 1 &  Pregunta 2 &     Total\\
 \hline

  \vspace{1.5cm} & &       \\
 \hline
 \end{tabular}
 \end{center}

Enviar documentos solicitados en el formato solicitado al correo 
\textbf{veranonumerico@gmail.com} .
%que el ayudante le indique.
\begin{enumerate}
\item (20 pt.) Implemente en un rutero llamado \texttt{newton.m} el algoritmo de descenso de Newton con \texttt{7} iteraciones para encontrar las ra\'ices de 
$$
f(x)= \frac{(x^3+1)sin(x^2+1)}{x^4+1}
$$
que est\'an a la izquierda y derecha de $x=-1$.

\textquestiondown Cu\'al es el valor las ra\'ices encontradas?. (10 pt.)

\fbox{ \begin{minipage}{\textwidth}   \hfill\vspace{1cm}   \end{minipage} } 

Adjunte el rutero \texttt{newton.m} al correo.

\textbf{Desarrollo:} C\'odigo \texttt{newton.m} \fbox{10 pt.}

\begin{lstlisting}
clear all; close all; clc
% Variable simbolica
syms x;
f=(x^3+1)*sin(x^2+1)./(x^4+1);
Df=diff(f,x);
% La funcion de la que buscamos su raiz y su derivada
f=inline(f);
Df=inline(Df);
%Graficamos la funcion
ezplot(f,[-3,3]); hold all;
plot(-1,f(-1),'*r')
axis on;

%% Metodo de Newton
% Para la raiz a la izquierda del -1
k=0;
raiz=-1.3;
while k<=7
    k=k+1;
    xk=raiz;
    fxk=feval(f,xk);
    Dfxk=feval(Df,xk);
    if (Dfxk==0)
        error('La derivada de la funcion se anula.')
    end
    corr=fxk/Dfxk;
    raiz=xk-corr;
    iter=k;
    plot(xk,f(xk),'*g');
end
disp(['El valor encontrado es ', num2str(xk)]);

% Para la raiz a la derecha del -1
k=0;
raiz=1;
while k<=7
    k=k+1;
    xk=raiz;
    fxk=feval(f,xk);
    Dfxk=feval(Df,xk);
    if (Dfxk==0)
        error('La derivada de la funcion se anula.')
    end
    corr=fxk/Dfxk;
    raiz=xk-corr;
    iter=k;
    plot(xk,f(xk),'*g');
end
disp(['El valor encontrado es ', num2str(xk)]);
\end{lstlisting}

\textquestiondown Cu\'al es el valor las ra\'ices encontradas?. (10 pt.)

\fbox{ \begin{minipage}{\textwidth}   
-1.46 y 1.46 \fbox{10pt.}
\hfill\vspace{1cm}   \end{minipage} } 


\item (20 pt.)
En un rutero llamado \texttt{sistema.m} determine si el sistema de ecuaciones
$$
\begin{array}{lrc}
x_1+3x_2+x_7	& =0	& \\
x_6+x_7			& =1	& \\
x_{i}+2x_{i-1}+3x_{i+1} &= i & \forall i\in\{2,3,4,5,6\}
\end{array}
$$
tiene \'unica soluci\'on y en tal caso, en el mismo rutero, encuentre su soluci\'on y escr\'ibala a continuaci\'on

\fbox{ \begin{minipage}{\textwidth}   \hfill

El sistema \fbox{ \begin{minipage}{0.5cm}   \hfill
\vspace{0.75cm} 
\end{minipage} } 
tiene \'unica soluci\'on

\vspace{0.25cm}
$x_1=$ \hspace{1cm},
$x_2=$ \hspace{1cm},
$x_3=$ \hspace{1cm},
$x_4=$ \hspace{1cm},
$x_5=$ \hspace{1cm},
$x_6=$ \hspace{1cm},
$x_7=$ \hspace{1cm}.

\vspace{0.25cm} 
\end{minipage} } 

Adjunte el rutero \texttt{sistema.m} al correo.

\textbf{Desarrollo:} C\'odigo de \texttt{sistema.m} \fbox{10pt.}
\begin{lstlisting}
%% Ensamblado de la matriz de coeficientes
B=   [diag(2*ones(5,1)),zeros(5,2)]...
    +[zeros(5,1),diag(1*ones(5,1)),zeros(5,1)]...
    +[zeros(5,2),diag(3*ones(5,1))];
A=[ 1,3,zeros(1,4),1    ;...
    zeros(1,5),1,1      ;...
    B];
b=[0,1,2:6]';

%% Resolucion del sistema
%El sistema es cuadrado entonces como
det(A)
%es distinto de cero, tiene unica solucion.
x=A\b
\end{lstlisting}

El sistema \fbox{ \begin{minipage}{0.5cm} SI  
\end{minipage} } 
tiene \'unica soluci\'on: \fbox{10pt.}

\begin{lstlisting}
>> x'

ans =

    3.2222   -1.5040   -0.9802    2.3294    1.2103   -0.2897    1.2897
\end{lstlisting}


\end{enumerate}
\end{document}   