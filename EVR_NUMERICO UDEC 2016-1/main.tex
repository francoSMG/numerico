\documentclass[11pt]{article}

\usepackage{multicol}
\usepackage{graphicx}
\usepackage[english]{babel}
\usepackage[utf8]{inputenc}
\usepackage{amsmath}
\usepackage{amssymb}
\usepackage{listings}
\usepackage[numbered,framed]{matlab-prettifier}
\lstMakeShortInline"
\lstset{
  style              = Matlab-editor,
  %basicstyle         = \mlttfamily,
  escapechar         = ",
  mlshowsectionrules = true,
}

\newcommand{\buena}{[BUENA]}
\newcommand{\justifique}{[JUSTIFIQUE]}
\newcommand{\nojustifique}{[NOJUSTIFIQUE]}

\begin{document}
\title{EVR_NUMERICO UDEC 2016-1}
\begin{itemize}

\item Considerando los puntos 
$$
\{(0,1/3),(1,1),(-1,1/3)\},
$$
la funci\'on del tipo 
$$
p(x)=ax^2+bx+c
$$
que mejor se ajusta a dichos puntos satisface
\begin{enumerate} 
\item $a=\frac{1}{3}$, $b=0$ y $c=\frac{2}{3}$.
\item $a=\frac{1}{3}$, $b=1$ y $c=\frac{2}{3}$.
\item $a=\frac{-1}{3}$, $b=0$ y $c=\frac{-4}{3}$.
\item \buena $a=\frac{-1}{3}$, $b=0$ y $c=\frac{4}{3}$.
\end{enumerate}
\justifique 

\item Para solucionar num\'ericamente el sistema de E.D.O.'s
$$
\begin{array}{rl|}
y''+2y'	&=cos(x)\\
x'+y'	&=sen(x)\\\hline
\end{array}
$$
se requiere de 
\begin{enumerate}
	\item Transformar el sistema en uno de cinco ecuaciones diferenciales.
    \item Transformar el sistema en uno de cuatro ecuaciones diferenciales.
    \item Transformar el sistema en uno de tres ecuaciones diferenciales.
    \item Transformar el sistema en uno de orden uno.
\end{enumerate}
Es correcto afirmar
\begin{enumerate}
	\item Sólo i) y ii) son verdaderas.
    \item Sólo i) y iii) son verdaderas.
    \item \buena Sólo iii) y iv) son verdaderas.
    \item Sólo iii) es verdaderas.
\end{enumerate}
\justifique 

\item Se establecen las siguientes proposiciones
\begin{enumerate}
	\item Todo sistema de dos E.D.O.'s de orden dos se transforma en un sistema de cuatro E.D.O's de orden 1.
    \item El método RK45 tiene un error global de orden 4.
    \item El método de Adams-Bashfoth puede solucionar cualquier P.V.I.
\end{enumerate}
Decida cual de las siguientes opciones es correcta
\begin{enumerate}
	\item Sólo i) es falsa.
    \item Sólo i) y ii) son falsas.
    \item i), ii) y iii) son falsas.
    \item \buena Sólo iii) es falsa.
\end{enumerate}
\nojustifique 

\item Decida cu\'al de las siguientes proposiciones es \textbf{falsa} y justifique su elecci\'on mostrando un contraejemplo.
\begin{enumerate}
	\item Existen m\'etodos num\'ericos para resolver P.V.I. de orden 5.
    \item \buena Toda soluci\'on num\'erica de un P.V.I. generada por un m\'etodo Runge-Kutta tiene error global mayor que cero.
    \item Existen integrales que se no se pueden calcular exactamente con la regla del punto medio.
    \item El método de Newton permite encontrar ra\'ices de un sistema de ecuaciones no lineales.
\end{enumerate}
\justifique

\end{itemize}
\end{document}