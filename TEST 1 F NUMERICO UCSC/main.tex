\documentclass[11pt]{article}

\usepackage[english]{babel}
\usepackage[utf8]{inputenc}
\usepackage{amsmath}
\usepackage{amssymb}
\usepackage{graphicx}
\usepackage[colorinlistoftodos]{todonotes}
\usepackage{listings,multicol}
\usepackage{textcomp}
\usepackage{hyperref}

\setlength{\oddsidemargin}{0.5cm} \setlength{\evensidemargin}{0cm}
\setlength{\textwidth}{16cm} \setlength{\textheight}{23cm}
\setlength{\topmargin}{-0.5cm}
\textheight 21.5cm


\begin{document}

\title{TEST 1 F NUMERICO UCSC}

\begin{minipage}{0.15\textwidth}
\includegraphics[width=\textwidth]{ucsc.png}
\end{minipage}
\begin{minipage}{0.9\textwidth}
{UNIVERSIDAD CAT\'OLICA}\\ 
{DE LA SANT\'ISIMA CONCEPCI\'ON}\\
{DEPARTAMENTO DE MATEM\'ATICA}\\ 
{ Y F\'ISICA APLICADAS}\\
\rule{0.66\textwidth}{.5pt} Franco A. Milanese
\end{minipage}

\vspace*{0.5cm} \centerline {\bf\underline{Test 1 formativo C\'alculo Num\'erico IN1012C }}
\centerline{\textrm{Semana 30 de marzo 2015}}  

\vspace{0.2cm}
\textbf{Nombre:} \hspace{0.65\textwidth}\textbf{Carrera:}

\vspace{0.1cm}
\textbf{Profesor de C\'atedra:}\hspace{0.5\textwidth} \textbf{ RUT:}
 \begin{center}
 \begin{tabular}{||p{2cm}|p{2cm}|p{2cm}||}
 \hline
 Pregunta 1 &  Pregunta 2 &     Total\\
 \hline

  \vspace{1.5cm} & &       \\
 \hline
 \end{tabular}
 \end{center}

Enviar documentos solicitados en el formato solicitado al correo que el ayudante le indique.

\begin{enumerate}
\item (40 pt) Siga las siguientes instrucciones.
\begin{enumerate}
\item En un rutero llamado \texttt{test1p1.m} cree una matriz de orden $3\times 3$ que en sus filas tenga las coordenadas de los puntos 
$$
\{(0,0,0),(6.1,0,0),(6.1,6.1,0)\},
$$
estos puntos representan v\'ertices de un cubo.
\item A continuaci\'on en el rutero \texttt{test1p1.m}, utilizando el comando \texttt{plot3()}, grafique este cubo.
\item A continuaci\'on en el rutero \texttt{test1p1.m}, en una variable llamada \texttt{AREA} grabe el \'area del cubo.
\item A continuaci\'on en el rutero \texttt{test1p1.m}, en una variable llamada \texttt{VOLUMEN} grabe el volumen del cubo.
\end{enumerate}
Adjunte el rutero \texttt{test1p1.m} al correo.

\item Cree una funci\'on llamada \texttt{muestracubo()}, cuya entrada sean tres puntos de un cubo en el espacio y su salida sean la superficie y volumen utilizado por el cubo, y adem\'as que en el proceso genere una gr\'afica de tal cubo.
\end{enumerate}
\end{document}  