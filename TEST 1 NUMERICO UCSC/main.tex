\documentclass[11pt]{article}

\usepackage[english]{babel}
\usepackage[utf8]{inputenc}
\usepackage{amsmath}
\usepackage{amssymb}
\usepackage{graphicx}
\usepackage[colorinlistoftodos]{todonotes}
\usepackage{listings,multicol}
\usepackage{textcomp}
\usepackage{hyperref}

\setlength{\oddsidemargin}{0.5cm} \setlength{\evensidemargin}{0cm}
\setlength{\textwidth}{16cm} \setlength{\textheight}{23cm}
\setlength{\topmargin}{-0.5cm}
\textheight 21.5cm


\usepackage[numbered,framed]{matlab-prettifier}
\lstMakeShortInline"
\lstset{
  style              = Matlab-editor,
  %basicstyle         = \mlttfamily,
  escapechar         = ",
  mlshowsectionrules = true,
}


\begin{document}

\title{TEST 1 NUMERICO UCSC}

\begin{minipage}{0.15\textwidth}
\includegraphics[width=\textwidth]{ucsc.png}
\end{minipage}
\begin{minipage}{0.9\textwidth}
{UNIVERSIDAD CAT\'OLICA}\\ 
{DE LA SANT\'ISIMA CONCEPCI\'ON}\\
{DEPARTAMENTO DE MATEM\'ATICA}\\ 
{ Y F\'ISICA APLICADAS}\\
\rule{0.66\textwidth}{.5pt} Franco A. Milanese
\end{minipage}

\vspace*{0.5cm} \centerline {\bf\underline{Test 1 C\'alculo Num\'erico IN1012C }}
\centerline{\textrm{Martes 14 de abril de 2015}}  

\vspace{0.2cm}
\textbf{Nombre:} PAUTA \hspace{0.65\textwidth}\textbf{Carrera:}

\vspace{0.1cm}
\textbf{Profesor de C\'atedra:}\hspace{0.5\textwidth} \textbf{ RUT:}
 \begin{center}
 \begin{tabular}{||p{2cm}|p{2cm}|p{2cm}||}
 \hline
 Pregunta 1 &  Pregunta 2 &     Total\\
 \hline

  \vspace{1.5cm} & &       \\
 \hline
 \end{tabular}
 \end{center}

Enviar documentos solicitados en el formato solicitado al correo 
\textbf{veranonumerico@gmail.com} .
%que el ayudante le indique.
\begin{enumerate}
\item (20 pt) Descargue el archivo ubicado en 
\begin{center}
\url{http://www.udec.cl/~fmilanese/Indultos2006-2012.csv}
\end{center}
este contiene el detalle de los indultos concedidos a criminales entre los a\~nos 2006 y 2012 en Chile. 
\begin{enumerate}
	\item En un rutero llamado \texttt{indultos.m} lleve a Matlab el archivo descargado usando el comando \texttt{dlmread()}. 
    \item En el mismo rutero grafique la cantidad de indultos otorgados por a\~no.
    \item Grabe la gr\'afica generada como \texttt{indultosporanho.jpg}
\end{enumerate}
Adjunte el programa \texttt{indultos.m} y la gr\'afica \texttt{indultosporanho.jpg} al correo.

\textquestiondown C\'ual es el crimen con mayor cantidad de indultos?.

\fbox{ \begin{minipage}{\textwidth}  

TR\'AFICO IL\'ICITO DE ESTUPEFACIENTES 
\hfill\vspace{1cm} 
\end{minipage} } 



\textbf{Desarrollo: } 
\begin{enumerate}
\item El fichero descargado es  
\begin{verbatim}
ANHO,SEXO,EDAD,DELITO,MOTIVO
2006,M,26,HOMICIDIO SIMPLE + ROBO CON INTIMIDACION + MALTRATO DE OBRA A CARABINEROS,RAZONES HUMANITARIAS 
2006,F,35,TRAFICO ILICITO DE ESTUPEFACIENTES,RAZONES HUMANITARIAS 
2006,M,47,ROBO CON FUERZA EN LUGAR NO HABITADO (FRUSTRADO),
2006,M,42,BIGAMIA,SITUACION FAMILIAR
2006,M,31,ROBO CON INTIMIDACION,POSIBILIDAD DE POSTULACION A BENEFICIOS
2006,M,32,ROBO CON INTIMIDACION,REINSERCION SOCIAL
2006,F,39,TRAFICO ILICITO DE ESTUPEFACIENTES,RAZONES ECONOMICAS
2006,F,50,TRAFICO ILICITO DE ESTUPEFACIENTES,RAZONES ECONàMICAS
2006,M,24,ROBO CON FUERZA EN LUGAR HABITADO,REINSERCION SOCIAL Y LABORAL
2006,M,46,HURTO SIMPLE + DESACATO,RAZONES ECONOMICAS
2006,M,58,ROBO CON INTIMIDACION,RAZONES HUMANITARIAS 
2006,M,55,GIRO DOLOSO DE CHEQUE,RAZONES HUMANITARIAS
2006,M,53,TRAFICO ILICITO DE ESTUPEFACIENTES,RAZONES ECONOMICAS
2006,F,34,TRAFICO ILICITO DE ESTUPEFACIENTES,RAZONES ECONOMICAS
2006,F,46,TRAFICO ILICITO DE ESTUPEFACIENTES,RAZONES ECONOMICAS
2006,M,49,TRAFICO ILICITO DE ESTUPEFACIENTES,RAZONES ECONOMICAS
2006,M,28,INFRACCION LEY DE ARMAS,RAZONES ECONOMICAS
2006,M,56,TRAFICO ILICITO DE ESTUPEFACIENTES,RAZONES ECONOMICAS
2006,F,72,TRAFICO ILICITO DE ESTUPEFACIENTES,RAZONES ECONOMICAS
2006,F,50,TRAFICO ILICITO DE ESTUPEFACIENTES,RAZONES HUMANITARIAS 
2006,M,39,TRAFICO ILICITO DE ESTUPEFACIENTES,RAZONES ECONOMICAS
2006,M,48,TRAFICO ILICITO DE ESTUPEFACIENTES,RAZONES ECONOMICAS
2006,M,42,CUASIDELITO CON RESULTADO MULTIPLE DE HOMICIDIO Y LESIONES GRAVES,RAZONES LABORALES
2006,M,30,TRAFICO ILICITO DE ESTUPEFACIENTES,RAZONES ECONOMICAS
2006,M,65,TRAFICO ILICITO DE ESTUPEFACIENTES,RAZONES ECONOMICAS
2006,F,29,TRAFICO ILICITO DE ESTUPEFACIENTES,POSTULACION A BENEFICIOS
2006,F,57,TRAFICO ILICITO DE ESTUPEFACIENTES,RAZONES ECONOMICAS
2006,M,50,ROBO CON VIOLENCIA,RAZONES ECONOMICAS
2006,M,58,TRAFICO ILICITO DE ESTUPEFACIENTES,RAZONES HUMANITARIAS 
2006,F,46,INFRACCION LEY N 19.366,RAZONES HUMANITARIAS
2006,M,37,LESIONES MENOS GRAVES,RAZONES ECONOMICAS
2006,F,36,TRAFICO ILICITO DE ESTUPEFACIENTES,RAZONES ECONOMICAS
2006,F,56,TRAFICO ILICITO DE ESTUPEFACIENTES,RAZONES ECONOMICAS
2006,M,45,TRAFICO ILICITO DE ESTUPEFACIENTES,RAZONES ECONOMICAS
2006,M,56,ROBO CON FUERZA EN LUGAR DESTINADO A LA HABITACIàN (2),RAZONES HUMANITARIAS 
2006,M,33,INFRACCION LEY N 18.290,RAZONES LABORALES
2006,M,31,ROBO CON FUERZA EN LUGAR DESTINADO A LA HABITACIàN ,POSTULACION A BENEFICIOS
2006,M,43,ROBO CON INTIMIDACION,RAZONES ECONOMICAS
2006,M,28,MANEJO EN ESTADO DE EBRIEDAD,POSTULACION A BENEFICIOS
2007,M,69,VIOLACION + INCESTO,RAZONES HUMANITARIAS
2007,M,38,MANEJO EN ESTADO DE EBRIEDAD CAUSANDO LESIONES,RAZONES LABORALES
2007,M,50,GIRO DOLOSO DE CHEQUES (9),RAZONES ECONOMICAS
2007,M,38,LESIONES GRAVES + ROBO CON FUERZA + ROBO CON VIOLENCIA (2),RAZONES HUMANITARIAS
2007,F,49,MANEJO EN ESTADO DE EBRIEDAD  ,RAZONES LABORALES
2007,M,55,USURPACION DE IDENTIDAD + ESTAFA EN GRADO FRUSTRADO + USO MALICIOSO DE INSTRUMENTO PUBLICO FALSO + ROBO CON VIOLENCIA FRUSTRADO + PORTE Y TENENCIA ILEGAL DE ARMA DE FUEGO,RAZONES HUMANITARIAS 
2008,M,30,MANEJO EN ESTADO DE EBRIEDAD,RAZONES LABORALES
2008,M,42,ESTAFA,RAZONES LABORALES
2008,M,30,HURTO + ROBO CON INTIMIDACION,RAZONES HUMANITARIAS
2008,M,29,ROBO CON VIOLENCIA,RAZONES HUMANITARIAS
2008,M,61,TRAFICO ILICITO DE ESTUPEFACIENTES,RAZONES HUMANITARIAS 
2008,M,71,VIOLACION,RAZONES HUMANITARIAS 
2008,F,37,ROBO CON INTIMIDACION,RAZONES HUMANITARIAS 
2008,M,25,ROBO CON VIOLENCIA,RAZONES HUMANITARIAS 
2008,M,24,ROBO POR SORPRESA,RAZONES HUMANITARIAS
2008,M,42,ROBO CON INTIMIDACION,RAZONES HUMANITARIAS 
2008,F,33,TRAFICO ILICITO DE ESTUPEFACIENTES,RAZONES HUMANITARIAS 
2008,M,53,PORTE ILEGAL DE ARMA DE FUEGO + ROBO CON INTIMIDACIàN + QUEBRANTAMIENTO,RAZONES HUMANITARIAS 
2008,M,25,ROBO CON INTIMIDACION,RAZONES HUMANITARIAS
2009,M,44,ROBO CON VIOLENCIA,RAZONES HUMANITARIAS 
2009,M,54,MANEJO EN ESTADO DE EBRIEDAD,RAZONES LABORALES
2009,M,45,PORTE ILEGAL DE ARMA DE FUEGO,RAZONES HUMANITARIAS 
2009,M,55,TENENCIA ILEGAL DE ARMA DE FUEGO (2) + TRµFICO ILÖCITO DE ESTUPEFACIENTES EN PEQUE¥AS CANTIDADES,RAZONES HUMANITARIAS 
2009,M,54,TRAFICO ILICITO DE ESTUPEFACIENTES,RAZONES HUMANITARIAS 
2009,M,49,ROBO CON FUERZA EN LUGAR HABITADO + ROBO CON VIOLENCIA + PORTE Y TENENCIA ILEGAL DE ARMA DE FUEGO,RAZONES HUMANITARIAS 
2009,M,56,HOMICIDIO SIMPLE,RAZONES HUMANITARIAS 
2009,M,67,TRAFICO ILICITO DE ESTUPEFACIENTES,RAZONES HUMANITARIAS 
2009,M,29,ROBO CON FUERZA EN LUGAR HABITADO (2),RAZONES HUMANITARIAS 
2010,F,36,TRAFICO ILICITO DE ESTUPEFACIENTES,RAZONES HUMANITARIAS
2010,M,32,ROBO EN BIENES NACIONALES DE USO PéBLICO + ROBO POR SORPRESA + ROBO CON FUERZA EN LUGAR DESTINADO A LA HABITACIàN,RAZONES HUMANITARIAS 
2010,M,65,FALSIFICACION DE INSTRUMENTO PéBLICO (REITERADO) + FALSIFICACION DE INSTRUMENTO PRIVADO (REITERADO) + MALVERSACION DE CAUDALES PUBLICOS (REITERADO) + APROPIACION INDEBIDA (REITERADO),RAZONES HUMANITARIAS
2011,M,62,LESIONES GRAVES + HURTO + LESIONES LEVES (FALTA),RAZONES HUMANITARIAS 
2011,M,43,ROBO CON FUERZA EN BIEN NACIONAL DE USO PéBLICO + ROBO CON VIOLENCIA,RAZONES HUMANITARIAS 
2011,M,24,ROBO CON INTIMIDACION,RAZONES HUMANITARIAS 
2011,M,43,TRAFICO ILICITO DE ESTUPEFACIENTES,RAZONES HUMANITARIAS 
2012,M,52,LESIONES GRAVES EN CONTEXTO DE VIOLENCIA INTRAFAMILIAR (2),RAZONES HUMANITARIAS 
2012,F,29,ABANDONO DE MENOR DE DIEZ ANHOS EN LUGAR SOLITARIO,
2012,M,59,HOMICIDIO CALIFICADO + AMENAZAS NO CONDICIONALES,RAZONES HUMANITARIAS
2012,M,21,ROBO CON VIOLENCIA,RAZONES HUMANITARIAS
2012,M,70,TRAFICO ILICITO DE ESTUPEFACIENTES,RAZONES HUMANITARIAS
2012,M,53,TRAFICO ILICITO DE ESTUPEFACIENTES,RAZONES HUMANITARIAS
\end{verbatim}
el cual no se puede leer con \texttt{dlmread()} puesto contiene car\'acteres que no son n\'umeros. Lo modificamos elminando su primera fila y  segunda  y\'ultima columna.
\begin{verbatim}
2006;26
2006;35
2006;47
2006;42
2006;31
2006;32
2006;39
2006;50
2006;24
2006;46
2006;58
2006;55
2006;53
2006;34
2006;46
2006;49
2006;28
2006;56
2006;72
2006;50
2006;39
2006;48
2006;42
2006;30
2006;65
2006;29
2006;57
2006;50
2006;58
2006;46
2006;37
2006;36
2006;56
2006;45
2006;56
2006;33
2006;31
2006;43
2006;28
2007;69
2007;38
2007;50
2007;38
2007;49
2007;55
2008;30
2008;42
2008;30
2008;29
2008;61
2008;71
2008;37
2008;25
2008;24
2008;42
2008;33
2008;53
2008;25
2009;44
2009;54
2009;45
2009;55
2009;54
2009;49
2009;56
2009;67
2009;29
2010;36
2010;32
2010;65
2011;62
2011;43
2011;24
2011;43
2012;52
2012;29
2012;59
2012;21
2012;70
2012;53
\end{verbatim}
luego ejecutamos su lectura con el c\'odigo 
\begin{lstlisting}
DATA=dlmread('Indultos2006-2012c.csv');
\end{lstlisting}
\item Para graficar los datos debemos agrupar los indulos por año y graficar seg\'un el c\'odigo
\begin{lstlisting}
DATA=dlmread('Indultos2006-2012c.csv');
ANHOS=2006:1:2012;
IND2006=length(DATA(find(DATA(:,1)==ANHOS(1)),1));
IND2007=length(DATA(find(DATA(:,1)==ANHOS(2)),1));
IND2008=length(DATA(find(DATA(:,1)==ANHOS(3)),1));
IND2009=length(DATA(find(DATA(:,1)==ANHOS(4)),1));
IND2010=length(DATA(find(DATA(:,1)==ANHOS(5)),1));
IND2011=length(DATA(find(DATA(:,1)==ANHOS(6)),1));
IND2012=length(DATA(find(DATA(:,1)==ANHOS(7)),1));
plot(ANHOS,[IND2006,IND2007,IND2008,IND2009,IND2010,IND2011,IND2012]);
\end{lstlisting}
lo que genera la gr\'afica

\includegraphics[width=\textwidth]{indultosporanho.jpg}


\end{enumerate}

\item (40 pt) Considere la funci\'on
$$
f(x)=
\begin{cases}
\frac{cos(x)}{x} \quad  \text{, si } x\in\,[-5,0[\\
\frac{sin(x)}{x} \quad  \text{, si } x\in\,]0,5[
\end{cases}.
$$
En una misma figura grafique las funciones dadas por $f(x)$, $f(f(x))$, $f(x+1)$ y $f(x^2+1)$. Grabe esta im\'agen como \texttt{funciones.jpg} y adj\'untela al correo.

\textbf{Desarrollo:}

Debido a al composici\'on nos conviene usar un fichero tipo \texttt{function} para trabajar este problema, creamos
\begin{lstlisting}
function y=f(x)
for i=1:length(x)
	if(x(i)>=-5 && x(i)<0)
    	y(i)=cos(x(i))/x(i);
    elseif (x(i)>=0 && x(i)<5)
       y(i)=sin(x(i))/x(i);
    end
end
end
\end{lstlisting}
y graficamos con el rutero
\begin{lstlisting}
x=linspace(-5,4.9,123);
figure(1);
subplot(2,2,1);
plot(x,f(x));
subplot(2,2,2);
plot(x,f(f(x)));
subplot(2,2,3);
x=-6:0.1:3.99;
plot(x,f(x+1));
subplot(2,2,4);
x=-sqrt(4):0.1:sqrt(3.9);
plot(x,f(x.^2+1));
\end{lstlisting}
lo que genera la gr\'afica solicitada

\includegraphics[width=\textwidth]{funciones.jpg}

\end{enumerate}
\end{document}   